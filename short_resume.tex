%%%%%%%%%%%%%%%%%%%%%%%%%%%%%%%%%%%%%%%%%
% Medium Length Professional CV
% LaTeX Template
% Version 2.0 (8/5/13)
%
% This template has been downloaded from:
% http://www.LaTeXTemplates.com
%
% Original author:
% Trey Hunner (http://www.treyhunner.com/)
%
% Important note:
% This template requires the resume.cls file to be in the same directory as the
% .tex file. The resume.cls file provides the resume style used for structuring the
% document.
%
%%%%%%%%%%%%%%%%%%%%%%%%%%%%%%%%%%%%%%%%%

%----------------------------------------------------------------------------------------
%	PACKAGES AND OTHER DOCUMENT CONFIGURATIONS
%----------------------------------------------------------------------------------------

\documentclass{short_resume} % Use the custom resume.cls style
\usepackage[dvipsnames]{xcolor}

\usepackage{gensymb}

\usepackage[left=0.75in,top=0.6in,right=0.75in,bottom=0.1in]{geometry} % Document margins
\newcommand{\tab}[1]{\hspace{.2667\textwidth}\rlap{#1}}
\newcommand{\itab}[1]{\hspace{0em}\rlap{#1}}
\name{Jeremy K. Thaller} % Your name
\address{10 Knowlton Dr. \\ Acton, MA 01720} % Your address
%\address{123 Pleasant Lane \\ City, State 12345} % Your secondary addess (optional)
\address{github: loisks317 \\ blog: loisks.blogspot.com}
\address{978-496-7990 \\ jkt2@williams.edu} % Your phone number and email
%\definecolor{darkpurple}{RGB}{108,48,130}

\renewenvironment{rSection}[1]{
	\sectionskip
	\textcolor{RoyalPurple}{\MakeUppercase{#1}}
	\sectionlineskip
	\hrule
	\begin{list}{}{
			\setlength{\leftmargin}{1.5em}
		}
		\item[]
	}{
	\end{list}
}



\begin{document}
	
	%----------------------------------------------------------------------------------------
	%	EDUCATION SECTION
	%----------------------------------------------------------------------------------------
	
	\begin{rSection}{Education}
		{\bf Technische Universit{\"a}t M{\"u}nchen (TUM)} \hfill {\em September 2019 - September 2020} 
		\\Anticipated M.S. in Applied and Engineering Physics
		\\Erasmus Mundus: Masters in Materials Science Exploring Large Scale Facilities
		
		{\bf Williams College} \hfill {\em September 2015 - Present} 
		\\(\emph{in progress}) B.A. in Physics with Honors\\
		Pre-engineering studies \hfill
		
		
		{\bf Acton-Boxborough Regional High School} \hfill {\em 2011-2015} 
		\\ National AP Scholar \hfill
		\\ National Honors Society

		
		
	\end{rSection}
	%----------------------------------------------------------------------------------------
	%	TECHNICAL STRENGTHS SECTION
	%----------------------------------------------------------------------------------------
	\newcommand{\CC}{C\nolinebreak\hspace{-.05em}\raisebox{.4ex}{\tiny\bf +}\nolinebreak\hspace{-.10em}\raisebox{.4ex}{\tiny\bf +}}
	\def\CC{{C\nolinebreak[4]\hspace{-.05em}\raisebox{.4ex}{\tiny\bf ++}}}
	
	\begin{rSection}{Technical Strengths}
		
		\begin{tabular}{ @{} >{\bfseries}l @{\hspace{6ex}} l }
			Programming Languages &  MATLAB, JAVA, HTML, Python, Arduino (C/\CC) \\
			Data Software & Mathematica, Excel, LabView, LoggerPro \\
			Other Software & LaTeX, Solid Works, Adobe Illustrator, Adobe Photoshop   \\
			Machining Experience & Bridgeport Milling, CNC Milling, 3D Printing, Laser Cutting \\
		\end{tabular}
		
	\end{rSection}
	
	%----------------------------------------------------------------------------------------
	%	RESEARCH EXPERIENCE SECTION
	%----------------------------------------------------------------------------------------

	
	\begin{rSection}{Research Experience}
		
		\begin{rSubsection}{Soft Condensed Matter Physics}{May 2018 - Present}{Undergraduate Honors Thesis}{}
			\vspace{-.5em}
				\item[] {\em Advised by Katharine E. Jensen, Professor of Physics}\hfill {\em Williams College}
				\item Designed and built stretching apparatus to induce equibiaxial stretch in soft materials
				\item Used Fluorescent Confocal Microscopy to measure the strain dependency of solid surface stress is soft materials via adhesion
				\item Data was collected through modified MATLAB scripts from K.E. Jensen and M.L. Kilfoil
		
		\end{rSubsection}
%		
		%----------------------------------------------------------------------------


		\begin{rSubsection}{Atomic, Molecular, and Optical Physics}{June - August 2017}{Undergraduate Research Assistant}{}
			\vspace{-.5em}
			\item[] {\em Advised by Protik K. Majumder, Professor of Physics}\hfill {\em Williams College}
			\item Took data towards an ultra-precise  measurement of the Electric Quadrupole (E2) amplitude within the $6S^26P^2$ $^3P_0$ $\rightarrow$ $^3P_2$ transition in Pb
			\item Programed a PID controller in LabView to thermally regulate an oven to within $\pm .4\degree$ C at temperatures around $ 950\degree $ C
			\item Designed a deposition-rate detector for an indium cell chamber based on the mass dependent frequency of Quartz Crystals
		\end{rSubsection}

\end{rSection}

\pagebreak

%------------------------------------------------------------------------------------
%       Coursework
%------------------------------------------------------------------------------------	
	\begin{rSection}{Advanced Physics Coursework} \itemsep -2pt
		Condensed Matter Physics (Spring 2019) \\
		Thermodynamics and Statistical Mechanics (Spring 2019)\\
		Advanced Classical Mechanics and Fluid Dynamics (Tutorial) \\
		Gravity \\
		Particle Physics (Tutorial)\\
		Quantum Mechanics \\
		Philosophical Implications of Modern Physics \\
		Electricity and Magnetism \\
		Mathematical Methods for Scientists \\
		Vibrations, Waves, and Optics
	\end{rSection}




	
	
	
	
%------------------------------------------------------------------------------------
%       Additional Information
%------------------------------------------------------------------------------------	
	\begin{rSection}{Additional Information} \itemsep -2pt
		\begin{tabular}{ @{} >{\bfseries}l @{\hspace{6ex}} l }
			Interests &  Bassoon, Jazz Piano, Running, Bicycle Repair, Arduinos, Rocketry, Graphic Design \\
			Languages &  German (Currently B2)
		\end{tabular}
	\end{rSection}




\end{document}


