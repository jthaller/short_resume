%%%%%%%%%%%%%%%%%%%%%%%%%%%%%%%%%%%%%%%%%
% This template has been downloaded from:
% http://www.LaTeXTemplates.com
%\\
% Original author:
% Trey Hunner (http://www.treyhunner.com/)
%
% Important note:
% This template requires the resume.cls file to be in the same directory as the
% .tex file. The resume.cls file provides the resume style used for structuring the
% document.
%
%%%%%%%%%%%%%%%%%%%%%%%%%%%%%%%%%%%%%%%%%

%----------------------------------------------------------------------------------------
%	PACKAGES AND OTHER DOCUMENT CONFIGURATIONS
%----------------------------------------------------------------------------------------

\documentclass{short_resume} % Use the custom resume.cls style
\usepackage[dvipsnames]{xcolor}
\usepackage{gensymb}
\usepackage{graphicx}
\usepackage{hyperref}
\hypersetup{
	colorlinks=true, %set true if you want colored links
	linktoc=all,     %set to all if you want both sections and subsections linked
	allcolors = black %get rid of weird auto link highlighting
}
\usepackage{gensymb} %need for \degree
\usepackage[left=0.5in,top=0.4in,right=0.4in,bottom=0.4in]{geometry} % Document margins
\newcommand{\tab}[1]{\hspace{.2667\textwidth}\rlap{#1}}
\newcommand{\itab}[1]{\hspace{0em}\rlap{#1}}
\name{Jeremy K. Thaller} % Your name
\address{\href{mailto:thallerjeremy@gmail.com}{thallerjeremy@gmail.com} \\ \href{https://jthaller.github.io/portfolio}{jthaller.github.io/portfolio} \\ \href{https://github.com/jthaller}{github.com/jthaller}} % Your address


\renewenvironment{rSection}[1]{
	\sectionskip
	\textcolor{RoyalPurple}{\MakeUppercase{#1}}
	\sectionlineskip
	\hrule
	\begin{list}{}{
			\setlength{\leftmargin}{1.5em}
		}
		\item[]
	}{
	\end{list}
}

\begin{document}
%	\vspace{-.75em}
%	\begin{rSubsection}{}{\hspace{-2em}\textbf{tl;dr: }\textit{US citizen; Aug. 2021 MSci. graduate with ML experience; looking for entry level data science roles}}{}{}
%	\item[] \vspace{-.75em}
%	\end{rSubsection}	
	\vspace{-1em}
	\hfill \textbf{ ~~~~~About: }\textit{US citizen; Aug.~2021 MSci.~graduate with significant python and ML experience} \hfill
	\vspace{-1em}
	
	%----------------------------------------------------------------------------------------
	%	EDUCATION SECTION
	%----------------------------------------------------------------------------------------
	\begin{rSection}{Education} 
		\begin{rSubsection}{Ludwig Maximilians \& Technische Universit{\"a}t M{\"u}nchen}{2019 -- Aug. 2021}{}{}
			\vspace{-.2em}
	%		MSci. in Geomaterials and Geochemistry
			\item[] {(Anticipated) Joint MSci. in Materials Science and Engineering}\hfill{\em Munich, Germany}
			\end{rSubsection}
		\vspace{-.4em}
		\begin{rSubsection}{Adam Mickiwicz University}{2019 -- Aug. 2021}{}{}
			\vspace{-.2em}
%			MSci. in Physics of Advanced Materials for Energy Processing
			\item[] {(Anticipated) MSci. in Applied Physics}\hfill{\em Poznan, Poland}
		\end{rSubsection}
	\vspace{-.5em}
		\begin{rSubsection}{Williams College}{\em 2015 -- 2019}{}{}
			\vspace{-.2em}
			\item[] {B.A. in Physics with Honors $\vert$ GPA 3.3}\hfill{\em Williamstown, MA}
			\item[] Sigma Xi Honors Society Inductee $\vert$ Captain of Track \& Field Team	
		\end{rSubsection}
	\end{rSection}
		
\vspace{-1.75em}		
	%----------------------------------------------------------------------------------------
	%   WORK EXPERIENCE
	%----------------------------------------------------------------------------------------
	\begin{rSection}{Work Experience}
			\begin{rSubsection}{Brookhaven National Lab}{Feb. -- Aug. 2021}{Master's Thesis Researcher, Structure and Dynamics of Applied Nanomaterials Group}{}
			\item[] Applying supervised machine learning to particle accelerator data to determine nanoparticle disorder from x-ray absorption near edge (XANES) spectroscopy spectra
			\vspace{-.2em}
		\end{rSubsection}
		\vspace{-.2em}
		\begin{rSubsection}{Yale University}{Summer 2019}{Postbac Researcher, Department of Mechanical Engineering and Materials Science}{}
%			\vspace{-.5em}
%			\item[] {\em Advised by Jan Schroers, Professor of Physics}\hfill {\em Yale University}
			\item[] Nanomolded crystalline metals and analyzed the samples with electron microscopy; wrote and deployed a python program to quicken repetitive calculations 
			\vspace{-.2em}
		\end{rSubsection}
		\vspace{-.2em}
		\begin{rSubsection}{Williams College}{2018 -- 2019}{Undergraduate Thesis Researcher, Department of Physics}{}
%			\vspace{-.5em}
%			\item[] {\em Advised by Katharine E. Jensen, Professor of Physics}\hfill {\em Williams College}
			\item[] Designed and CNC milled a stretching apparatus for soft materials; collected data via confocal microscopy, processed and analyzed data with MATLAB and multinomial linear regressions to measure the strain dependency of solid surface stress in silicone gel.
		\end{rSubsection}
		\vspace{-.2em}
		\begin{rSubsection}{Williams College}{Summer 2017}{Undergraduate Research Assistant, Department of Physics}{}
%			\vspace{-.5em}
%			\item[] {\em Advised by Protik K. Majumder, Professor of Physics}\hfill {\em Williams College}
			\item[] Programmed and installed a PID controller to regulate furnace temperatures within 1~\degree C, designed a deposition-rate detector, and analyzed absorption spectrum data with MATLAB and Mathematica
		\end{rSubsection}

%		\begin{rSubsection}{Student Technology Assistant}{Summer 2016}{Office of Information Technology}{Williams College}
%		%			\vspace{-.5em}
%		%			\item[] {\em Advised by Protik K. Majumder, Professor of Physics}\hfill {\em Williams College}
%		\item[] Installed and imaged new computers, worked the student and faculty support help-lines, repaired damaged computers
%		\vspace{-.2em}
%		\end{rSubsection}
	\end{rSection}
		
	\vspace{-1.75em}
%----------------------------------------------------------------------------------------
%	TECHNICAL STRENGTHS SECTION
%----------------------------------------------------------------------------------------
\newcommand{\CC}{C\nolinebreak\hspace{-.05em}\raisebox{.4ex}{\tiny\bf +}\nolinebreak\hspace{-.10em}\raisebox{.4ex}{\tiny\bf +}}
\def\CC{{C\nolinebreak[4]\hspace{-.05em}\raisebox{.4ex}{\tiny\bf ++}}}

\begin{rSection}{Technical Strengths}
	
	\begin{tabular}{ @{} >{\bfseries}l @{\hspace{6ex}} l }
		Programming Languages &  Python, MATLAB, SQL, Java, Arduino (\CC~Variant)\\
		Python Packages & Pandas, NumPy, Scikit-Learn, PyTorch, TensorFlow, KERAS, Optuna  \\
		 & Tensorboard,  Seaborn, Matplotlib, Plotly/Dash \\
		Data Software & Mathematica, Quantum Espresso, Excel, LabView, LoggerPro \\
		Visualization Software & LaTeX, Solid Works, VESTA, Adobe Illustrator, Adobe Photoshop
	\end{tabular}
	
\end{rSection}

\vspace{-1.5em}
%------------------------------------------------------------------------------------
%       Recent Projects
%------------------------------------------------------------------------------------
\begin{rSection}{Recent Projects} \itemsep -2pt
		\textbf{Facebook Messenger Analysis} \\
		HTML and JSON scraping, 
		statistics analysis and visualization,
		mystery friend classifier, 
		and chatbot \vspace{.5em}
		\\
		\textbf{Organic Semiconductor Optimization Predictor}\\
		Neural network to reduce semiconductor candidate screening time by $ 100\times $
\end{rSection}

\vspace{-1.5em}

%------------------------------------------------------------------------------------
%       Data Science Skills
%------------------------------------------------------------------------------------
\begin{rSection}{Data Science Skills} \itemsep -2pt
	\begin{tabular}{ @{} >{}l @{\hspace{6ex}} l }
		Data Cleaning and Feature Engineering, 
		SSH + VIM,
		Unix Command Line (BASH), 
		Git and Version Control, \\
		Probability and Statistics (Bayesian),
		Neural Networks and Deep Learning,
		Natural Language Processing, \\ 
		Image Classification and Computer Vision, 
		Recommendation Systems
	\end{tabular}
\end{rSection}


% %------------------------------------------------------------------------------------
% %       Advanced Physics Coursework
% %------------------------------------------------------------------------------------
% 	\begin{rSection}{Relevant Coursework} \itemsep -2pt
% 	\begin{tabular}{ @{} >{}l @{\hspace{6ex}} l }
% 		Computational Materials Design &  Thermodynamics and Statistical Mechanics  \\
% 		Molecular Dynamic Simulations & Neutron Diffraction and Structural Determination \\
% 		Advanced Classical Mechanics and Fluid Dynamics  & Particle Physics (Oxbridge Style Tutorial)\\
% 	\end{tabular}
% \end{rSection}

%	Computational Materials Design \hfill $\diamond$ \hfill Molecular Dynamic Simulations \hfill $\diamond$  \hfill Solid State Physics \hfill $\diamond$  \hfill Gravity $\diamond$\\
%\hfill Advanced Classical Mechanics \& Fluid Mechanics \hfill $\diamond$ \hfill Particle Physics \hfill $\diamond$ \hfill Statistical Mechanics \hfill $ \diamond $ \hfill Neutron Diffraction  $\diamond$ Linear Algebra \hfill $\diamond$ \hfill Multivariate and Vector Calculus \hfill $ \diamond $ \hfill Neutron Diffraction \hfill $ \diamond $
%------------------------------------------------------------------------------------
%       DOCUMENT END
%------------------------------------------------------------------------------------	
\end{document}
